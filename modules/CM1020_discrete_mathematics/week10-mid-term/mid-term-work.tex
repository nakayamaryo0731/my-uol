\documentclass{article}
\usepackage{amsmath,amssymb}
\usepackage{enumitem}

\begin{document}

\section*{Question 1}

\subsection*{(a)}

Let 
\[
X = \{2, 4, 6, 8, 10\}, \quad Y = \{4, 6, 8, 10, 12\}, \quad Z = \{2, 10, 12, 14, 16\}
\]

\begin{enumerate}[label=\roman*.]

\item Find $X \cap Y \cap Z$ and $X \cup (Y \setminus Z)$.
\[
\begin{aligned}
X \cap Y &= \{4, 6, 8, 10\} \\
X \cap Y \cap Z &= \{10\} \quad \text{(only element common to all three sets)} \\
Y \setminus Z &= \{4, 6, 8\} \\
X \cup (Y \setminus Z) &= \{2, 4, 6, 8, 10\} \cup \{4, 6, 8\} = \{2, 4, 6, 8, 10\}
\end{aligned}
\]

\item Is $X \subseteq X \cap Y \cap Z$? Justify your answer.

\[
\begin{aligned}
X &= \{2, 4, 6, 8, 10\}, \quad X \cap Y \cap Z = \{10\} \\
\text{Not all elements of } X \text{ are in } X \cap Y \cap Z, \text{ so } X \subseteq X \cap Y \cap Z \text{ is false.}
\end{aligned}
\]

\item Find the number of elements in the power set $\mathcal{P}(X \cup (Y \setminus Z))$.

\[
\begin{aligned}
X \cup (Y \setminus Z) &= \{2, 4, 6, 8, 10\} \\
\text{Number of elements} &= 5 \\
\text{Size of the power set} &= 2^5 = \boxed{32}
\end{aligned}
\]

\end{enumerate}

\subsection*{(b)}

Determine all sets $A$ which satisfy:
\[
\mathcal{P}(A) \subseteq \{\emptyset, \{\emptyset\}, \{\{\emptyset\}\}\}
\]

We are given that the power set of $A$ must be a subset of the set:
\[
\{\emptyset, \{\emptyset\}, \{\{\emptyset\}\}\}
\]
This set contains three distinct elements:
- $\emptyset$
- $\{\emptyset\}$
- $\{\{\emptyset\}\}$

We must now find all sets $A$ such that every element of $\mathcal{P}(A)$ is contained within this set.  
We refer to this set as the \textbf{RHS} (right-hand side).

Try all possible subsets of the RHS as candidates for $A$:

\begin{itemize}
  \item $A = \emptyset$\\
  $\mathcal{P}(A) = \{\emptyset\}$, which is a subset of the RHS.

  \item $A = \{\emptyset\}$\\
  $\mathcal{P}(A) = \{\emptyset, \{\emptyset\}\}$, which is a subset of the RHS.

  \item $A = \{\{\emptyset\}\}$\\
  $\mathcal{P}(A) = \{\emptyset, \{\{\emptyset\}\}\}$, which is a subset of the RHS.

  \item $A = \{\emptyset, \{\emptyset\}\}$\\
  $\mathcal{P}(A)$ contains $\{\emptyset, \{\emptyset\}\}$, which is not in the RHS.

  \item $A = \{\emptyset, \{\{\emptyset\}\}\}$\\
  $\mathcal{P}(A)$ contains $\{\emptyset, \{\{\emptyset\}\}\}$, which is not in the RHS.

  \item $A = \{\{\emptyset\}, \{\emptyset\}\}$\\
  $\mathcal{P}(A)$ contains $\{\{\emptyset\}, \{\emptyset\}\}$, which is not in the RHS.

  \item $A = \{\emptyset, \{\emptyset\}, \{\{\emptyset\}\}\}$\\
  $\mathcal{P}(A)$ has $2^3 = 8$ elements, most of which are not in the RHS.
\end{itemize}

\textbf{Final Answer:}
\[
\boxed{
A = \emptyset,\quad
A = \{\emptyset\},\quad
A = \{\{\emptyset\}\}
}
\]

\subsection*{(c)}

Let \( A, B, C \) be sets. We are asked to determine whether the following logical equivalence is true:
\[
A \subseteq B \quad \text{if and only if} \quad A \cap C \subseteq B \cap C \quad \text{for all sets } C.
\]

We examine both directions of the equivalence.

\paragraph{(\( \Rightarrow \)) Direction:} Assume that \( A \subseteq B \). We want to show that for any set \( C \), it follows that
\[
A \cap C \subseteq B \cap C.
\]

Let \( x \in A \cap C \). Then \( x \in A \) and \( x \in C \). Since \( A \subseteq B \), we also have \( x \in B \). Therefore, \( x \in B \cap C \).  
So \( A \cap C \subseteq B \cap C \), which proves this direction.

\paragraph{(\( \Leftarrow \)) Direction:} Now suppose that
\[
A \cap C \subseteq B \cap C \quad \text{for all sets } C.
\]
We want to show that this implies \( A \subseteq B \).  
We will show that this is \textbf{not always true} by providing a counterexample.

Let
\[
A = \{1\}, \quad B = \{2\}, \quad C = \{1\}.
\]
Then,
\[
A \cap C = \{1\}, \quad B \cap C = \emptyset.
\]
So \( A \cap C \nsubseteq B \cap C \), and the condition fails.  
But to make the implication fail even when the condition holds, try this:

Let
\[
A = \{1\}, \quad B = \emptyset, \quad C = \{2\}.
\]
Then
\[
A \cap C = \emptyset, \quad B \cap C = \emptyset,
\]
so \( A \cap C \subseteq B \cap C \) is true. But \( A \not\subseteq B \).  
Therefore, the condition
\[
A \cap C \subseteq B \cap C \text{ for all } C
\]
does not imply \( A \subseteq B \).

\paragraph{Conclusion:}  
The implication
\[
A \subseteq B \Rightarrow A \cap C \subseteq B \cap C \text{ for all } C
\]
is always true, but the converse is not.  
So the logical equivalence
\[
A \subseteq B \iff A \cap C \subseteq B \cap C \text{ for all } C
\]
is \textbf{not valid in general}.

\subsection*{(d)}

Let \( A, B, C \) be subsets of a universal set \( U \). We are given the condition:
\[
A \subseteq B \quad \text{and} \quad C \subseteq \overline{B}
\]
and we are asked whether this implies:
\[
A \cap C = \emptyset.
\]

\paragraph{Proof:}

Let \( x \in A \cap C \). Then by definition of intersection,
\[
x \in A \quad \text{and} \quad x \in C.
\]
Since \( A \subseteq B \), it follows that \( x \in B \).  
Since \( C \subseteq \overline{B} \), it follows that \( x \notin B \).  

This leads to a contradiction: \( x \in B \) and \( x \notin B \).  
Therefore, our initial assumption that such an \( x \) exists must be false.  
Hence,
\[
A \cap C = \emptyset.
\]

\paragraph{Conclusion:}  
The statement is logically valid. If \( A \subseteq B \) and \( C \subseteq \overline{B} \), then \( A \cap C = \emptyset \).

\section*{Question 2}

\subsection*{(a)}

Determine whether each of the following expressions defines a function. If not, explain why.

\begin{enumerate}[label=\roman*.]

\item \( f : \mathbb{R} \to \mathbb{R}, \quad f(x) = \dfrac{1}{\ln(x - 1)} \)

This expression is undefined when \( \ln(x - 1) = 0 \) or when the logarithm is undefined.

\[
\ln(x - 1) \text{ is undefined for } x \leq 1,\quad \text{and } \ln(x - 1) = 0 \text{ when } x = e
\]

So \( f(x) \) is undefined for \( x \leq 1 \), and also \( f(x) = \dfrac{1}{0} \) is undefined at \( x = e \approx 2.718 \)

Hence, the function is \textbf{not defined on all of } \( \mathbb{R} \), so it is \textbf{not a valid function from } \( \mathbb{R} \to \mathbb{R} \).

\textbf{Conclusion:} Not a function as defined, because \( f(x) \) is undefined for some \( x \in \mathbb{R} \).

\item \( f : \mathbb{Z} \to \mathbb{Z}, \quad f(x) = x^3 - 2x + 5 \)

This is a polynomial function of degree 3 with integer coefficients.  
For any \( x \in \mathbb{Z} \), \( f(x) \in \mathbb{Z} \).

\textbf{Conclusion:} This is a function from \( \mathbb{Z} \to \mathbb{Z} \).

\item \( f : \mathbb{R} \to \mathbb{R}, \quad f(x) = \sqrt{x - 3} \)

The square root function \( \sqrt{x - 3} \) is defined only when \( x - 3 \geq 0 \), i.e., \( x \geq 3 \)

Therefore, \( f(x) \) is undefined for \( x < 3 \), so the function is not defined on all of \( \mathbb{R} \)

\textbf{Conclusion:} Not a function from \( \mathbb{R} \to \mathbb{R} \), because the domain must exclude values \( x < 3 \)

\end{enumerate}

\subsection*{(b)}

Given
\[
f(x) = x^2 + b \quad \text{and} \quad g(x) = \sqrt{x + 3}
\]
we are asked to find all values of \( b \) such that
\[
(f \circ g)(x) = (g \circ f)(x)
\]
for all \( x \) in the domain.

\paragraph{Step 1: Compute \( (f \circ g)(x) \)}
\[
(f \circ g)(x) = f(g(x)) = f(\sqrt{x + 3}) = (\sqrt{x + 3})^2 + b = x + 3 + b
\]

\paragraph{Step 2: Compute \( (g \circ f)(x) \)}
\[
(g \circ f)(x) = g(f(x)) = g(x^2 + b) = \sqrt{x^2 + b + 3}
\]

\paragraph{Step 3: Equate both expressions}
\[
x + 3 + b = \sqrt{x^2 + b + 3}
\]

Now square both sides:
\[
(x + 3 + b)^2 = x^2 + b + 3
\]

Expand the left-hand side:
\[
(x + 3 + b)^2 = x^2 + 6x + 9 + 2bx + 6b + b^2
\]
\[
= x^2 + 2bx + 6x + b^2 + 6b + 9
\]

Now set it equal to the right-hand side:
\[
x^2 + 2bx + 6x + b^2 + 6b + 9 = x^2 + b + 3
\]

Subtract \( x^2 \) from both sides:
\[
2bx + 6x + b^2 + 6b + 9 = b + 3
\]

This must hold for all \( x \), so the coefficient of \( x \) must be zero:
\[
2b + 6 = 0 \quad \Rightarrow \quad b = -3
\]

Now substitute \( b = -3 \) into the constant terms:
\[
b^2 + 6b + 9 = (-3)^2 + 6(-3) + 9 = 9 - 18 + 9 = 0
\]
and RHS:
\[
b + 3 = -3 + 3 = 0
\]

\textbf{Conclusion:} The only possible value is
\[
\boxed{b = -3}
\]

\subsection*{(c)}

Solve the equation:
\[
2\log_4(x) - \log_4(3x - 2) = 0
\]

\paragraph{Step 1: Use logarithmic identities}

Recall that \( a \log_b(x) = \log_b(x^a) \), so
\[
2\log_4(x) = \log_4(x^2)
\]

Now substitute:
\[
\log_4(x^2) - \log_4(3x - 2) = 0
\]

Use the identity \( \log_b(a) - \log_b(b) = \log_b\left(\frac{a}{b}\right) \):
\[
\log_4\left( \frac{x^2}{3x - 2} \right) = 0
\]

\paragraph{Step 2: Exponentiate both sides}

\[
\frac{x^2}{3x - 2} = 4^0 = 1
\]

\paragraph{Step 3: Solve the equation}

\[
x^2 = 3x - 2 \\
x^2 - 3x + 2 = 0 \\
(x - 1)(x - 2) = 0
\]

So the solutions are:
\[
x = 1 \quad \text{or} \quad x = 2
\]

\paragraph{Step 4: Check for validity in the domain}

We must check that all logarithmic expressions are defined and positive:

- For \( x = 1 \):
  - \( \log_4(1) \) is defined
  - \( 3(1) - 2 = 1 > 0 \Rightarrow \log_4(1) \) is defined

- For \( x = 2 \):
  - \( \log_4(2) \) is defined
  - \( 3(2) - 2 = 4 > 0 \Rightarrow \log_4(4) \) is defined

\textbf{Conclusion:} Both values are valid.

\[
\boxed{x = 1 \quad \text{or} \quad x = 2}
\]

\subsection*{(d)}

Let \( f : \mathbb{R}^* \to \mathbb{R} \) be defined by
\[
f(x) = e^x + x,
\]
where \( \mathbb{R}^* = \mathbb{R} \setminus \{0\} \).

\begin{enumerate}[label=\roman*.]

\item Determine whether or not \( f \) is a one-to-one function.

\textbf{Method: Use the definition of injectivity.}

Suppose \( f(x_1) = f(x_2) \), then:
\[
e^{x_1} + x_1 = e^{x_2} + x_2.
\]

We want to show this implies \( x_1 = x_2 \).  
Assume instead that \( x_1 \neq x_2 \). Then, without loss of generality, suppose \( x_1 < x_2 \).  
Since the exponential function \( e^x \) is strictly increasing and \( x \) is linear, the sum \( e^x + x \) is also strictly increasing.

Therefore:
\[
x_1 < x_2 \Rightarrow f(x_1) < f(x_2),
\]
which contradicts the assumption that \( f(x_1) = f(x_2) \).  
So our assumption was false, and we conclude that:
\[
\boxed{f \text{ is one-to-one}}
\]

\item Determine whether or not \( f \) is an onto function.

We must determine whether, for every \( y \in \mathbb{R} \), there exists \( x \in \mathbb{R}^* \) such that
\[
f(x) = y.
\]

Let \( y = f(x) = e^x + x \).  
We attempt to solve for \( x \) given any \( y \), but there is no algebraic expression to isolate \( x \) directly.  
Instead, examine the behavior of \( f(x) \):

\begin{itemize}
  \item As \( x \to -\infty \), \( e^x \to 0 \) and \( x \to -\infty \), so \( f(x) \to -\infty \)
  \item As \( x \to \infty \), \( f(x) \to \infty \)
  \item But \( x = 0 \notin \mathbb{R}^* \), and \( f(0) = 1 \) is not in the domain
\end{itemize}

Now consider the range of \( f(x) \) over \( \mathbb{R}^* \). Since \( f(x) \) is continuous and strictly increasing, its range is:
\[
(-\infty, 1) \cup (1, \infty)
\]
but does not include \( y = 1 \) (no \( x \in \mathbb{R}^* \) satisfies \( f(x) = 1 \)).

Hence,
\[
\boxed{f \text{ is not onto, since } 1 \notin \text{range of } f}
\]

\end{enumerate}

\subsection*{(e)}

Let \( f : A \to B \) and \( g : B \to C \) be functions.  
Assume that both \( f \) and \( g \) are one-to-one (injective).  
We aim to prove that the composition \( g \circ f : A \to C \) is also one-to-one.

\paragraph{Proof:}

To show that \( g \circ f \) is one-to-one, we must prove:
\[
(g \circ f)(x_1) = (g \circ f)(x_2) \Rightarrow x_1 = x_2.
\]

Let \( x_1, x_2 \in A \), and suppose:
\[
(g \circ f)(x_1) = (g \circ f)(x_2).
\]

By the definition of composition:
\[
g(f(x_1)) = g(f(x_2)).
\]

Since \( g \) is one-to-one, this implies:
\[
f(x_1) = f(x_2).
\]

Now apply the fact that \( f \) is also one-to-one:
\[
f(x_1) = f(x_2) \Rightarrow x_1 = x_2.
\]

Therefore, we have shown:
\[
(g \circ f)(x_1) = (g \circ f)(x_2) \Rightarrow x_1 = x_2,
\]
which means \( g \circ f \) is one-to-one.

\paragraph{Conclusion:}  
If both \( f \) and \( g \) are one-to-one, then their composition \( g \circ f \) is also one-to-one.
\[
\boxed{g \circ f \text{ is one-to-one.}}
\]

\section*{Question 3}

\subsection*{(a)}

Let \( p, q, r \) be propositions.

\begin{enumerate}[label=\roman*.]

\item Construct a truth table for the following compound propositions:

\[
(p \oplus q) \to r \qquad \text{and} \qquad (p \lor q) \to (p \land r)
\]

\begin{center}
\begin{tabular}{|c|c|c||c|c|c||c|c|}
\hline
$p$ & $q$ & $r$ & $p \oplus q$ & $(p \oplus q) \to r$ & $p \lor q$ & $p \land r$ & $(p \lor q) \to (p \land r)$ \\
\hline
T & T & T & F & T & T & T & T \\
T & T & F & F & T & T & F & F \\
T & F & T & T & T & T & T & T \\
T & F & F & T & F & T & F & F \\
F & T & T & T & T & T & F & F \\
F & T & F & T & F & T & F & F \\
F & F & T & F & T & F & F & T \\
F & F & F & F & T & F & F & T \\
\hline
\end{tabular}
\end{center}

\item Determine whether either of the above compound propositions is a tautology. Explain your answer.

\begin{itemize}
  \item The proposition \( (p \oplus q) \to r \) is false when \( p \oplus q = \text{T} \) and \( r = \text{F} \). This occurs in rows 4 and 6.
  \item Therefore, it is \textbf{not a tautology}.
  \item The proposition \( (p \lor q) \to (p \land r) \) is false when \( p \lor q = \text{T} \) and \( p \land r = \text{F} \). This occurs in rows 2, 4, 5, and 6.
  \item Therefore, it is also \textbf{not a tautology}.
\end{itemize}

\textbf{Conclusion:} Neither of the compound propositions is a tautology.
\end{enumerate}

\subsection*{(b)}

Let \( p, q, r, s \) be propositions. We are given:
\[
p = \text{F}, \quad q = \text{T}, \quad r = \text{F}, \quad s = \text{T}
\]

We are asked to find the truth value of the following proposition:

\[
((p \lor \neg r) \land (q \to s)) \leftrightarrow ((\neg p \land q) \lor (r \to s))
\]

\paragraph{Step 1: Evaluate each component}

\begin{itemize}
  \item \( \neg r = \neg \text{F} = \text{T} \)
  \item \( p \lor \neg r = \text{F} \lor \text{T} = \text{T} \)
  \item \( q \to s = \text{T} \to \text{T} = \text{T} \)
  \item Left side: \( \text{T} \land \text{T} = \text{T} \)
  \item \( \neg p = \text{T}, \quad \neg p \land q = \text{T} \land \text{T} = \text{T} \)
  \item \( r \to s = \text{F} \to \text{T} = \text{T} \)
  \item Right side: \( \text{T} \lor \text{T} = \text{T} \)
\end{itemize}

\paragraph{Step 2: Final result}

\[
\text{Left side} = \text{T}, \quad \text{Right side} = \text{T}
\Rightarrow \text{Entire expression} = \text{T}
\]

\paragraph{Conclusion:}
\[
\boxed{\text{The truth value of the compound proposition is true.}}
\]

\subsection*{(c)}

Let the propositions be defined as follows:

\begin{itemize}
  \item \( p \): "The employee completes the safety training."
  \item \( q \): "The employee passes the skills assessment."
  \item \( r \): "The employee qualifies for the fieldwork assignment."
\end{itemize}

Express the following compound propositions symbolically:

\begin{enumerate}[label=\roman*.]

\item \textit{“To qualify for the fieldwork assignment, it is necessary for the employee to both complete the safety training and pass the skills assessment.”}

This translates to:

\[
r \to (p \land q)
\]

\item \textit{“The employee qualifies for the fieldwork assignment only if they either complete the safety training or pass the skills assessment, but not both.”}

- "only if" means \( r \to \text{(rest)} \)
- "either ... or, but not both" means exclusive or, written as \( p \oplus q \)

So the full symbolic form is:

\[
r \to (p \oplus q)
\]

\item \textit{“The employee will qualify for the fieldwork assignment if, and only if, they complete the safety training and pass the skills assessment.”}

This is a biconditional statement:

\[
r \leftrightarrow (p \land q)
\]

\end{enumerate}

\subsection*{(d)}

Given the implication:

\[
\forall x \in \mathbb{R}, \quad \text{if } x^2 - 3x + 2 > 0 \text{ then } x > 2 \text{ or } x < 1
\]

We define:

\[
P(x): x^2 - 3x + 2 > 0, \quad Q(x): x > 2 \text{ or } x < 1
\]

The original implication is:

\[
P(x) \to Q(x)
\]

Now we write the following:

\begin{itemize}

\item \textbf{Contrapositive:}
\[
\forall x \in \mathbb{R}, \quad \text{if } \neg Q(x) \text{ then } \neg P(x)
\]
That is,
\[
\text{if } 1 \leq x \leq 2 \text{ then } x^2 - 3x + 2 \leq 0
\]

\item \textbf{Converse:}
\[
\forall x \in \mathbb{R}, \quad \text{if } Q(x) \text{ then } P(x)
\]
That is,
\[
\text{if } x > 2 \text{ or } x < 1 \text{ then } x^2 - 3x + 2 > 0
\]

\item \textbf{Inverse:}
\[
\forall x \in \mathbb{R}, \quad \text{if } \neg P(x) \text{ then } \neg Q(x)
\]
That is,
\[
\text{if } x^2 - 3x + 2 \leq 0 \text{ then } 1 \leq x \leq 2
\]

\end{itemize}

\subsection*{(e)}

We are asked to determine whether the following compound proposition is a tautology:
\[
((p \land q) \lor (r \to s)) \leftrightarrow ((p \lor r) \to s) \land ((q \lor r) \to s)
\]

\paragraph{Step 1: Rewriting using logical equivalences}

Recall the implication equivalence:
\[
a \to b \equiv \neg a \lor b
\]

Apply this to each implication:
\[
r \to s \equiv \neg r \lor s,\quad (p \lor r) \to s \equiv \neg(p \lor r) \lor s,\quad (q \lor r) \to s \equiv \neg(q \lor r) \lor s
\]

So the original expression becomes:
\[
((p \land q) \lor (\neg r \lor s)) \leftrightarrow ((\neg(p \lor r) \lor s) \land (\neg(q \lor r) \lor s))
\]

We now analyze whether this biconditional is always true.

\paragraph{Strategy:}

We consider two main cases based on the truth value of \( s \).

\medskip
\textbf{Case 1: \( s = \text{T} \)}

\begin{itemize}
  \item Left-hand side: \( (p \land q) \lor (\neg r \lor \text{T}) = \text{T} \)
  \item Right-hand side: both disjunctions end with \( \lor \text{T} = \text{T} \), so entire RHS is also \( \text{T} \)
\end{itemize}

Thus, when \( s = \text{T} \), the full expression is true.

\medskip
\textbf{Case 2: \( s = \text{F} \)}

We must analyze both sides more carefully.

\begin{itemize}
  \item Left-hand side: 
  \[
  (p \land q) \lor (\neg r \lor \text{F}) = (p \land q) \lor \neg r
  \]

  \item Right-hand side:
  \[
  (\neg(p \lor r) \lor \text{F}) \land (\neg(q \lor r) \lor \text{F}) = \neg(p \lor r) \land \neg(q \lor r)
  \]
  \[
  = \neg p \land \neg r \land \neg q \land \neg r = \neg p \land \neg q \land \neg r
  \]
\end{itemize}

Now try a specific assignment:

\[
p = \text{T},\quad q = \text{T},\quad r = \text{F},\quad s = \text{F}
\]

Then:

\begin{itemize}
  \item LHS: \( (T \land T) \lor \neg F = T \lor T = T \)
  \item RHS: \( \neg T \land \neg T \land \neg F = F \land F \land T = F \)
\end{itemize}

So:
\[
\text{LHS} = \text{T},\quad \text{RHS} = \text{F} \Rightarrow \text{Biconditional is false}
\]

\paragraph{Conclusion:}

We found an assignment of truth values for which the biconditional is false.  
Therefore, the proposition is not always true, so it is not a tautology.

\[
\boxed{\text{The proposition is not a tautology.}}
\]

\section*{Question 4}

\subsection*{(a)}

Translate the following statements into predicate logic using appropriate quantifiers and predicates.

\begin{enumerate}[label=\roman*.]

\item "Every doctor in the hospital wears a mask."

Let the domain be all people.  
Let \( D(x) \): "x is a doctor in the hospital"  
Let \( M(x) \): "x wears a mask"

\[
\forall x \, (D(x) \to M(x))
\]

\item "If a plant is watered regularly, then it grows healthy."

Let the domain be all plants.  
Let \( W(x) \): "x is watered regularly"  
Let \( H(x) \): "x grows healthy"

\[
\forall x \, (W(x) \to H(x))
\]

\item "There exists a number that is larger than any other number."

Let the domain be all real numbers.  
Let \( L(x, y) \): "x is larger than y"

\[
\exists x \, \forall y \, (x \neq y \to L(x, y))
\]

\item "All animals that have wings can fly."

Let the domain be all animals.  
Let \( W(x) \): "x has wings"  
Let \( F(x) \): "x can fly"

\[
\forall x \, ((W(x)) \to F(x))
\]

\end{enumerate}

\subsection*{(b)}

Determine whether each of the following statements is true or false. Justify your answer.

\begin{enumerate}[label=\roman*.]

\item \( \forall x \in \mathbb{Z}, \, \exists y \in \mathbb{R}^* \text{ such that } xy < 1 \)

\textbf{True.}  
Let \( x \in \mathbb{Z} \). We can always choose \( y = \frac{1}{2x} \) if \( x \neq 0 \), which satisfies \( xy = \frac{1}{2} < 1 \).  
If \( x = 0 \), choose any negative real \( y \) (e.g., \( y = -1 \)), then \( xy = 0 < 1 \).  
In all cases, a suitable nonzero real \( y \) exists such that \( xy < 1 \).

\[
\boxed{\text{True}}
\]

\item \( \forall x \in \mathbb{R}^*, \, \forall y \in \mathbb{Z} \text{ such that } xy > 1 \)

\textbf{False.}  
Choose \( x = 0.5 \in \mathbb{R}^* \) and \( y = 1 \in \mathbb{Z} \). Then \( xy = 0.5 < 1 \), which contradicts the condition.  
Since \( xy > 1 \) is not true for all \( x \in \mathbb{R}^*, y \in \mathbb{Z} \), the statement is false.

\[
\boxed{\text{False}}
\]

\item \( \forall x \in \mathbb{R}^*, \, \exists y \in \mathbb{R} \text{ such that } xy = 2 \)

\textbf{True.}  
Given any nonzero real number \( x \), we can choose \( y = \frac{2}{x} \), which is always defined in \( \mathbb{R} \) since \( x \neq 0 \).  
Then \( xy = x \cdot \frac{2}{x} = 2 \)

\[
\boxed{\text{True}}
\]

\end{enumerate}

\subsection*{(c)}

Negate the following quantified statement:

\[
\forall x \, \exists y \, (M(x) \land N(y)) \lor \forall z \, (K(z) \to L(z))
\]

\paragraph{Step 1: Apply negation to the full expression}

We negate the entire statement:

\[
\neg \left[ \forall x \, \exists y \, (M(x) \land N(y)) \lor \forall z \, (K(z) \to L(z)) \right]
\]

Use DeMorgan's Law:

\[
= \neg \left[ \forall x \, \exists y \, (M(x) \land N(y)) \right] \land \neg \left[ \forall z \, (K(z) \to L(z)) \right]
\]

---

\paragraph{Step 2: Push negation inside quantifiers}

Use standard logical equivalences:

\begin{itemize}
  \item \( \neg \forall x \, P(x) \equiv \exists x \, \neg P(x) \)
  \item \( \neg \exists y \, Q(y) \equiv \forall y \, \neg Q(y) \)
\end{itemize}

Apply to the first part:

\[
\neg \left[ \forall x \, \exists y \, (M(x) \land N(y)) \right] = \exists x \, \forall y \, \neg (M(x) \land N(y))
\]

Now apply DeMorgan inside the conjunction:

\[
\neg (M(x) \land N(y)) = \neg M(x) \lor \neg N(y)
\]

So the first part becomes:

\[
\exists x \, \forall y \, (\neg M(x) \lor \neg N(y))
\]

---

Now for the second part:

\[
\neg \left[ \forall z \, (K(z) \to L(z)) \right] = \exists z \, \neg (K(z) \to L(z))
\]

Recall:

\[
\neg (K(z) \to L(z)) = K(z) \land \neg L(z)
\]

So the second part becomes:

\[
\exists z \, (K(z) \land \neg L(z))
\]

---

\paragraph{Final Answer:}

\[
\boxed{
\exists x \, \forall y \, (\neg M(x) \lor \neg N(y)) \land \exists z \, (K(z) \land \neg L(z))
}
\]

\subsection*{(d)}

We are given the following premises:

\begin{align*}
1.\quad & s \to (p \lor q) \\
2.\quad & \neg p \to r \\
3.\quad & \neg q \to r \\
4.\quad & \neg s \\
\end{align*}

We are to determine whether the conclusion
\[
\therefore \ r
\]
logically follows from the premises.

\paragraph{Step 1: Use Modus Tollens on premise 1 and 4}

From premise 1: \( s \to (p \lor q) \)  
From premise 4: \( \neg s \)  

By \textbf{Modus Tollens}, this does not directly apply (since we would need \( \neg(p \lor q) \)), but we can use \textbf{Contrapositive}:

\[
s \to (p \lor q) \equiv \neg(p \lor q) \to \neg s
\]

Now we are given \( \neg s \), so from the contrapositive we can apply \textbf{Modus Ponens}:

\[
\neg(p \lor q) \text{ must be true}
\Rightarrow \neg p \land \neg q
\]

\paragraph{Step 2: Use Modus Ponens on premises 2 and 3}

From premise 2: \( \neg p \to r \), and we now know \( \neg p \) is true  
\[
\Rightarrow r
\]

Similarly, from premise 3: \( \neg q \to r \), and we know \( \neg q \) is true  
\[
\Rightarrow r
\]

So in both ways, we conclude \( r \) is true.

\paragraph{Conclusion:}

The conclusion \( r \) logically follows from the premises.  
Therefore, the argument is:
\[
\boxed{\text{Valid}}
\]

\subsection*{(a)}

Use DeMorgan's laws to simplify the following expressions:

\begin{enumerate}[label=\roman*.]

\item
\[
\overline{\, \overline{p \cdot \overline{q} \cdot r} + \overline{ \overline{r} \cdot s } \,}
\]

\textbf{Step 1: Apply DeMorgan's law to the outermost negation (OR to AND):}
\[
= \overline{ \overline{p \cdot \overline{q} \cdot r} } \cdot \overline{ \overline{ \overline{r} \cdot s } }
\]

\textbf{Step 2: Cancel double negations:}
\[
= (p \cdot \overline{q} \cdot r) \cdot (\overline{r} \cdot s)
\]

\textbf{Final simplified form:}
\[
\boxed{(p \cdot \overline{q} \cdot r) \cdot (\overline{r} \cdot s)}
\]

---

\item
\[
\overline{\, \overline{x} + y \,} \cdot \overline{\, x + \overline{y} \,} \cdot \overline{\, \overline{y} + \overline{z} \,}
\]

\textbf{Step 1: Apply DeMorgan's law to each term:}
\begin{align*}
\overline{\, \overline{x} + y \,} &= x \cdot \overline{y} \\
\overline{\, x + \overline{y} \,} &= \overline{x} \cdot y \\
\overline{\, \overline{y} + \overline{z} \,} &= y \cdot z
\end{align*}

\textbf{Step 2: Multiply all terms:}
\[
(x \cdot \overline{y}) \cdot (\overline{x} \cdot y) \cdot (y \cdot z)
\]

\textbf{Step 3: Group logically:}
\[
x \cdot \overline{y} \cdot \overline{x} \cdot y \cdot y \cdot z
= x \cdot \overline{x} \cdot y \cdot \overline{y} \cdot y \cdot z
\]

\textbf{Step 4: Use identity laws:}
\[
x \cdot \overline{x} = 0, \quad y \cdot \overline{y} = 0
\Rightarrow \text{entire expression equals } 0
\]

\textbf{Final simplified form:}
\[
\boxed{0}
\]

\end{enumerate}

\subsection*{(b)}

\begin{enumerate}[label=\roman*.]

\item \textbf{Find a Boolean expression for the output \( Q \) of the logic circuit.}

From the circuit:

- The top NAND gate takes \( A \) and \( \overline{B} \), so its output is:
  \[
  X = \overline{A \cdot \overline{B}}
  \]

- The bottom AND gate takes \( \overline{C} \) and \( D \), so its output is:
  \[
  Y = \overline{C} \cdot D
  \]

- The final NAND gate receives \( X \) and \( Y \), so the output is:
  \[
  Q = \overline{X \cdot Y} = \overline{ \left( \overline{A \cdot \overline{B}} \cdot ( \overline{C} \cdot D ) \right) }
  \]

\item \textbf{Simplify using Boolean algebra.}

Apply DeMorgan’s Law to the outermost NOT:
\[
Q = \overline{ \overline{A \cdot \overline{B}} } + \overline{ \overline{C} \cdot D }
\]

Now simplify each part:
\begin{align*}
\overline{ \overline{A \cdot \overline{B}} } &= A \cdot \overline{B} \\
\overline{ \overline{C} \cdot D } &= C + \overline{D}
\end{align*}

\textbf{Final simplified expression:}
\[
\boxed{Q = (A \cdot \overline{B}) + (C + \overline{D})}
\]

\end{enumerate}

\subsection*{(c)}

Use the duality principle to find the dual of the following Boolean equation:

\[
a \cdot b + c \cdot \overline{d} = (a + c) \cdot (a + \overline{d}) \cdot (b + c) \cdot (b + \overline{d})
\]

\textbf{Step: Apply the duality principle}

- Replace every \( \cdot \) (AND) with \( + \) (OR)
- Replace every \( + \) (OR) with \( \cdot \) (AND)
- Leave complements (negations) unchanged

\[
\text{LHS dual: } (a + b) \cdot (c + \overline{d})
\]
\[
\text{RHS dual: } (a \cdot c) + (a \cdot \overline{d}) + (b \cdot c) + (b \cdot \overline{d})
\]

\textbf{Final dual equation:}
\[
\boxed{
(a + b) \cdot (c + \overline{d}) = (a \cdot c) + (a \cdot \overline{d}) + (b \cdot c) + (b \cdot \overline{d})
}
\]

\subsection*{(d)}

Given the Boolean expression:
\[
F(A, B, C, D) = A \cdot \overline{B} \cdot C + A \cdot \overline{B} \cdot \overline{C} + \overline{A} \cdot B \cdot C \cdot \overline{D}
\]

\begin{enumerate}[label=\roman*.]

\item \textbf{Truth Table}

\begin{center}
\begin{tabular}{|c|c|c|c||c|}
\hline
$A$ & $B$ & $C$ & $D$ & $F(A,B,C,D)$ \\
\hline
F & F & F & F & F \\
F & F & F & T & F \\
F & F & T & F & F \\
F & F & T & T & F \\
F & T & F & F & F \\
F & T & F & T & F \\
F & T & T & F & \textbf{T} \\
F & T & T & T & F \\
T & F & F & F & \textbf{T} \\
T & F & F & T & \textbf{T} \\
T & F & T & F & \textbf{T} \\
T & F & T & T & \textbf{T} \\
T & T & F & F & F \\
T & T & F & T & F \\
T & T & T & F & F \\
T & T & T & T & F \\
\hline
\end{tabular}
\end{center}

Minterms where \( F = \text{T} \):  
\( m_6, m_8, m_9, m_{10}, m_{11} \)

---

\item \textbf{Karnaugh Map}

Use \( A, B \) for rows and \( C, D \) for columns. Row and column order in Gray code:

\[
\begin{array}{c|cccc}
AB \backslash CD & 00 & 01 & 11 & 10 \\
\hline
00 & F & F & F & F \\
01 & F & F & F & \textbf{T} \\
11 & F & F & F & F \\
10 & \textbf{T} & \textbf{T} & \textbf{T} & \textbf{T} \\
\end{array}
\]

---

\item \textbf{Minimal Sum of Products (SOP)}

From the K-map:

- Group of four T's on row \( A = \text{T}, B = \text{F} \):  
  \( A \cdot \overline{B} \)

- Single T at \( A = \text{F}, B = \text{T}, C = \text{T}, D = \text{F} \):  
  \( \overline{A} \cdot B \cdot C \cdot \overline{D} \)

\textbf{Final minimized SOP:}
\[
\boxed{F = A \cdot \overline{B} + \overline{A} \cdot B \cdot C \cdot \overline{D}}
\]

\end{enumerate}

\end{document}
